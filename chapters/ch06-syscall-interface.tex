\chapter{系统调用接口}

\section{系统设计}

系统调用接口是用户程序与操作系统内核交互的标准机制,它为用户程序访问系统服务提供了一套统一的API。
NimlothOS实现了完整的系统调用体系,支持文件操作、进程管理、信号处理等核心功能。

\subsection{调用约定}

NimlothOS遵循RISC-V架构的标准系统调用约定,使用\texttt{ecall}指令触发系统调用。
调用参数通过寄存器传递:系统调用号存储在\texttt{a7}寄存器,
参数按顺序存储在\texttt{a0}、\texttt{a1}、\texttt{a2}寄存器中,
返回值通过\texttt{a0}寄存器返回给用户程序。

\subsection{系统调用分类}

NimlothOS的系统调用按功能分为三大类:

\textbf{文件系统接口}提供了完整的文件I/O功能,包括文件的打开、关闭、读写操作,
以及管道通信机制,支持进程间的数据交换和重定向操作。

\textbf{进程管理接口}实现了完整的进程生命周期管理,支持进程创建、执行、等待和退出,
提供了fork-exec模型的完整实现,支持多进程并发执行和父子进程协作。

\textbf{信号处理接口}提供了异步事件处理机制,支持信号的发送、接收和自定义处理,
实现了类Unix的信号语义,为异常处理和进程间通信提供了支撑。

\subsection{安全机制}

系统调用的安全性通过多个层面保障。首先是\textbf{地址空间隔离},
所有用户空间指针都通过地址转换函数安全访问,防止用户程序访问内核内存。
其次是\textbf{参数验证},系统调用会检查所有输入参数的合法性,
包括文件描述符范围、内存地址有效性等。最后是\textbf{资源控制},
通过进程控制块管理用户程序的资源使用,防止资源泄漏和滥用。

\noindent
\rule{0.4\textwidth}{0.4pt}
\hfill
\text{以下为实现介绍}
\hfill
\rule{0.4\textwidth}{0.4pt}

\section{系统调用分发机制}

系统调用的处理始于陷阱处理程序,当用户程序执行\texttt{ecall}指令时,
硬件自动切换到内核态并跳转到陷阱处理入口。陷阱处理程序识别系统调用类型后,
调用\texttt{syscall}函数进行具体的系统调用分发。

\begin{lstlisting}[language=Rust,caption={系统调用分发器}, label={lst:syscall-dispatcher}]
pub fn syscall(syscall_id: usize, args: [usize; 3]) -> isize {
    match syscall_id {
        SYSCALL_READ => sys_read(args[0], args[1] as *const u8, args[2]),
        SYSCALL_WRITE => sys_write(args[0], args[1] as *const u8, args[2]),
        SYSCALL_EXIT => sys_exit(args[0] as i32),
        SYSCALL_YIELD => sys_yield(),
        SYSCALL_TIME => sys_time(),
        SYSCALL_PID => sys_pid(),
        SYSCALL_FORK => sys_fork(),
        SYSCALL_EXEC => sys_exec(args[0] as *const u8, args[1] as *const usize),
        SYSCALL_WAITPID => sys_waitpid(args[0] as isize, args[1] as *mut i32),
        SYSCALL_OPEN => sys_open(args[0] as *const u8, args[1] as u32),
        SYSCALL_CLOSE => sys_close(args[0]),
        SYSCALL_DUP => sys_dup(args[0]),
        SYSCALL_PIPE => sys_pipe(args[0] as *mut usize),
        SYSCALL_KILL => sys_kill(args[0], args[1] as i32),
        SYSCALL_SIGACTION => sys_sigaction(
            args[0] as i32,
            args[1] as *const SignalAction,
            args[2] as *mut SignalAction,
        ),
        SYSCALL_SIGPROCMASK => sys_sigprocmask(args[0] as u32),
        SYSCALL_SIGRETURN => sys_sigreturn(),
        _ => panic!("Unsupported syscall_id: {}", syscall_id),
    }
}
\end{lstlisting}

分发器采用模式匹配机制,根据系统调用号将请求路由到对应的处理函数。这样也便于扩展新的系统调用功能。
不支持的系统调用会导致系统panic,确保了接口的严格性和系统的稳定性。

\section{文件系统接口}

文件系统接口提供了标准的POSIX风格文件操作,支持文件的创建、读写、关闭等基本操作,
以及管道通信等高级功能。所有文件操作都基于文件描述符,提供了统一的I/O抽象。

\subsection{用户空间缓冲区访问}

由于用户程序和内核运行在不同的地址空间中,内核需要安全地访问用户空间的缓冲区数据。
NimlothOS通过\texttt{UserBuffer}抽象和地址转换函数来解决这个问题。

\texttt{translated\_byte\_buffer}函数将用户空间的缓冲区转换为内核可以安全访问的页面切片:

\begin{lstlisting}[language=Rust,caption={用户空间缓冲区转换}, label={lst:translated-buffer}]
pub fn translated_byte_buffer(token: usize, ptr: *const u8, len: usize) -> Vec<&'static mut [u8]> {
    let page_table = PageTable::from_token(token);
    let mut start = ptr as usize;
    let end = start + len;
    let mut v = Vec::new();
    while start < end {
        let start_va = VirtAddr::from(start);
        let mut vpn = start_va.floor();
        let ppn = page_table.translate(vpn).unwrap().ppn();
        vpn.step();
        let mut end_va: VirtAddr = vpn.into();
        end_va = end_va.min(VirtAddr::from(end));
        if end_va.page_offset() == 0 {
            v.push(&mut ppn.bytes_array()[start_va.page_offset()..]);
        } else {
            v.push(&mut ppn.bytes_array()[start_va.page_offset()..end_va.page_offset()]);
        }
        start = end_va.into();
    }
    v
}
\end{lstlisting}

\texttt{UserBuffer}结构封装了这些页面切片,提供了统一的读写接口:

\begin{lstlisting}[language=Rust,caption={UserBuffer结构}, label={lst:user-buffer}]
pub struct UserBuffer {
    pub buffers: Vec<&'static mut [u8]>,
}

impl UserBuffer {
    pub fn new(buffers: Vec<&'static mut [u8]>) -> Self {
        Self { buffers }
    }
    
    pub fn len(&self) -> usize {
        let mut total: usize = 0;
        for b in self.buffers.iter() {
            total += b.len();
        }
        total
    }
}
\end{lstlisting}

这样可以保障系统的\textbf{安全性},通过页表转换确保只访问用户程序拥有的内存页面;
并自动处理跨越多个物理页面的缓冲区;为文件系统提供统一的缓冲区抽象,无需关心底层的地址空间差异。

为支持文件系统的实际I/O操作,\texttt{UserBuffer}还实现了迭代器接口,
允许文件系统逐字节或逐块地访问用户数据:

\begin{lstlisting}[language=Rust,caption={UserBuffer迭代器实现}, label={lst:user-buffer-iter}]
impl IntoIterator for UserBuffer {
    type Item = *mut u8;
    type IntoIter = UserBufferIterator;
    
    fn into_iter(self) -> Self::IntoIter {
        UserBufferIterator {
            buffers: self.buffers,
            current_buffer: 0,
            current_idx: 0,
        }
    }
}

pub struct UserBufferIterator {
    buffers: Vec<&'static mut [u8]>,
    current_buffer: usize,
    current_idx: usize,
}

impl Iterator for UserBufferIterator {
    type Item = *mut u8;
    
    fn next(&mut self) -> Option<Self::Item> {
        if self.current_buffer >= self.buffers.len() {
            None
        } else {
            let r = &mut self.buffers[self.current_buffer][self.current_idx] as *mut _;
            if self.current_idx + 1 == self.buffers[self.current_buffer].len() {
                self.current_idx = 0;
                self.current_buffer += 1;
            } else {
                self.current_idx += 1;
            }
            Some(r)
        }
    }
}
\end{lstlisting}

它使得文件系统可以安全地遍历用户缓冲区中的每个字节,无论缓冲区是连续的还是跨越多个物理页面。

\subsection{文件读写操作}

\texttt{sys\_write}函数支持向文件描述符写入数据,通过\texttt{translated\_byte\_buffer}
安全地访问用户空间缓冲区:

\begin{lstlisting}[language=Rust,caption={文件写入系统调用}, label={lst:sys-write}]
pub fn sys_write(fd: usize, buf: *const u8, len: usize) -> isize {
    let token = current_user_token();
    let process = current_process().unwrap();
    let inner = process.inner_exclusive_access();
    if fd >= inner.fd_table.len() {
        return -1;
    }
    if let Some(file) = &inner.fd_table[fd] {
        if !file.writable() {
            return -1;
        }
        let file = file.clone();
        drop(inner);
        file.write(UserBuffer::new(translated_byte_buffer(token, buf, len))) as isize
    } else {
        -1
    }
}
\end{lstlisting}

对应的\texttt{sys\_read}函数提供读取功能:

\begin{lstlisting}[language=Rust,caption={文件读取系统调用}, label={lst:sys-read}]
pub fn sys_read(fd: usize, buf: *const u8, len: usize) -> isize {
    let token = current_user_token();
    let process = current_process().unwrap();
    let inner = process.inner_exclusive_access();
    if fd >= inner.fd_table.len() {
        return -1;
    }
    if let Some(file) = &inner.fd_table[fd] {
        if !file.readable() {
            return -1;
        }
        let file = file.clone();
        drop(inner);
        file.read(UserBuffer::new(translated_byte_buffer(token, buf, len))) as isize
    } else {
        -1
    }
}
\end{lstlisting}

\subsection{文件管理操作}

\texttt{sys\_open}支持多种打开模式,通过\texttt{translated\_str}安全地读取用户空间的文件路径:

\begin{lstlisting}[language=Rust,caption={文件打开系统调用}, label={lst:sys-open}]
pub fn sys_open(path: *const u8, flags: u32) -> isize {
    let process = current_process().unwrap();
    let token = current_user_token();
    let path = translated_str(token, path);
    if let Some(inode) = open_file(path.as_str(), OpenFlags::from_bits(flags).unwrap()) {
        let mut inner = process.inner_exclusive_access();
        let fd = inner.alloc_fd();
        inner.fd_table[fd] = Some(inode);
        fd as isize
    } else {
        -1
    }
}
\end{lstlisting}

\texttt{sys\_close}负责关闭文件描述符并释放相关资源:

\begin{lstlisting}[language=Rust,caption={文件关闭系统调用}, label={lst:sys-close}]
pub fn sys_close(fd: usize) -> isize {
    let process = current_process().unwrap();
    let mut inner = process.inner_exclusive_access();
    if fd >= inner.fd_table.len() {
        return -1;
    }
    if inner.fd_table[fd].is_none() {
        return -1;
    }
    inner.fd_table[fd].take();
    0
}
\end{lstlisting}

\subsection{文件描述符复制}

\texttt{sys\_dup}函数实现文件描述符的复制功能,新旧描述符共享同一个底层文件对象:

\begin{lstlisting}[language=Rust,caption={文件描述符复制}, label={lst:sys-dup}]
pub fn sys_dup(fd: usize) -> isize {
    let process = current_process().unwrap();
    let mut inner = process.inner_exclusive_access();
    if fd >= inner.fd_table.len() {
        return -1;
    }
    if inner.fd_table[fd].is_none() {
        return -1;
    }
    let new_fd = inner.alloc_fd();
    inner.fd_table[new_fd] = Some(Arc::clone(inner.fd_table[fd].as_ref().unwrap()));
    new_fd as isize
}
\end{lstlisting}

\subsection{管道通信机制}

\texttt{sys\_pipe}函数创建一对相互连接的文件描述符,支持进程间的单向数据传输:

\begin{lstlisting}[language=Rust,caption={管道创建系统调用}, label={lst:sys-pipe}]
pub fn sys_pipe(pipe: *mut usize) -> isize {
    let process = current_process().unwrap();
    let token = current_user_token();
    let mut inner = process.inner_exclusive_access();
    let (pipe_read, pipe_write) = make_pipe();
    let read_fd = inner.alloc_fd();
    inner.fd_table[read_fd] = Some(pipe_read);
    let write_fd = inner.alloc_fd();
    inner.fd_table[write_fd] = Some(pipe_write);
    *translated_refmut(token, pipe) = read_fd;
    *translated_refmut(token, unsafe { pipe.add(1) }) = write_fd;
    0
}
\end{lstlisting}

\section{进程管理接口}

进程管理接口实现了完整的进程生命周期管理,提供了进程创建、执行、等待、退出等核心功能,
支持多进程并发执行和父子进程协作机制。

\subsection{进程控制操作}

\texttt{sys\_exit}函数终止当前进程并设置退出码,会清理进程资源并调度下一个就绪进程运行:

\begin{lstlisting}[language=Rust,caption={进程退出系统调用}, label={lst:sys-exit}]
pub fn sys_exit(exit_code: i32) -> isize {
    println!("[kernel] Application exited with code {}", exit_code);
    exit_current_and_run_next(exit_code);
    panic!("Unreachable in sys_exit!");
}
\end{lstlisting}

\texttt{sys\_yield}函数实现协作式多进程,允许进程主动让出CPU时间片:

\begin{lstlisting}[language=Rust,caption={进程让权系统调用}, label={lst:sys-yield}]
pub fn sys_yield() -> isize {
    suspend_current_and_run_next();
    0
}
\end{lstlisting}

\subsection{进程信息获取}

\texttt{sys\_time}返回系统启动以来的毫秒数,\texttt{sys\_pid}返回当前进程的唯一标识符:

\begin{lstlisting}[language=Rust,caption={系统时间获取}, label={lst:sys-time}]
pub fn sys_time() -> isize {
    time_ms() as isize
}
\end{lstlisting}

\begin{lstlisting}[language=Rust,caption={进程ID获取}, label={lst:sys-pid}]
pub fn sys_pid() -> isize {
    current_process().unwrap().pid.0 as isize
}
\end{lstlisting}

\subsection{进程创建与替换}

\texttt{sys\_fork}函数创建当前进程的完整副本,父进程获得子进程PID,子进程返回值为0:

\begin{lstlisting}[language=Rust,caption={进程复制系统调用}, label={lst:sys-fork}]
pub fn sys_fork() -> isize {
    let current_process = current_process().unwrap();
    let new_process = current_process.fork();
    let new_pid = new_process.pid.0;
    let trap_cx = new_process.inner_exclusive_access().trap_cx();
    trap_cx.x[10] = 0; // x[10] = a0,设置子进程返回值为0
    add_process(new_process);
    new_pid as isize
}
\end{lstlisting}

\texttt{sys\_exec}函数用指定程序替换当前进程的地址空间,支持命令行参数传递:

\begin{lstlisting}[language=Rust,caption={程序执行系统调用}, label={lst:sys-exec}]
pub fn sys_exec(path: *const u8, mut args: *const usize) -> isize {
    let token = current_user_token();
    let path = translated_str(token, path);
    let mut args_vec = Vec::new();
    loop {
        let arg_str_ptr = *translated_ref(token, args);
        if arg_str_ptr == 0 {
            break;
        }
        args_vec.push(translated_str(token, arg_str_ptr as *const u8));
        unsafe {
            args = args.add(1);
        }
    }
    if let Some(data) = open_file(path.as_str(), OpenFlags::RDONLY) {
        let all_data = data.read_all();
        let process = current_process().unwrap();
        let argc = args_vec.len();
        process.exec(all_data.as_slice(), args_vec);
        argc as isize
    } else {
        -1
    }
}
\end{lstlisting}

\subsection{进程同步机制}

\texttt{sys\_waitpid}函数实现父子进程同步,支持等待特定PID的子进程或任意子进程结束:

\begin{lstlisting}[language=Rust,caption={进程等待系统调用}, label={lst:sys-waitpid}]
pub fn sys_waitpid(pid: isize, exit_code_ptr: *mut i32) -> isize {
    let process = current_process().unwrap();
    let mut inner = process.inner_exclusive_access();
    if !inner
        .children
        .iter()
        .any(|p| pid == -1 || pid as usize == p.getpid())
    {
        return -1;
    }
    let pair = inner.children.iter().enumerate().find(|(_, p)| {
        p.inner_exclusive_access().is_zombie() && (pid == -1 || pid as usize == p.getpid())
    });
    if let Some((idx, _)) = pair {
        let child = inner.children.remove(idx);
        let found_pid = child.getpid();
        let exit_code = child.inner_exclusive_access().exit_code;
        *translated_refmut(inner.memory_set.token(), exit_code_ptr) = exit_code;
        found_pid as isize
    } else {
        -2
    }
}
\end{lstlisting}

\section{信号处理接口}

信号处理接口提供了异步事件处理机制,支持进程间信号通信和异常处理,
实现了类Unix的信号语义,为系统异常处理和进程间通信提供了支持。

\subsection{信号发送机制}

\texttt{sys\_kill}函数支持向指定进程发送信号,会验证目标进程和信号的合法性:

\begin{lstlisting}[language=Rust,caption={信号发送系统调用}, label={lst:sys-kill}]
pub fn sys_kill(pid: usize, signum: i32) -> isize {
    if let Some(process) = pid2process(pid) {
        if let Some(flag) = SignalFlags::from_bits(1 << signum) {
            let mut process_ref = process.inner_exclusive_access();
            if process_ref.signals.contains(flag) {
                return -1; // 信号已存在
            }
            process_ref.signals.insert(flag);
            0
        } else {
            -1 // 非法信号编号
        }
    } else {
        -1 // 目标进程不存在
    }
}
\end{lstlisting}

\subsection{信号处理配置}

\texttt{sys\_sigaction}函数支持为指定信号安装自定义处理程序,禁止对关键信号的自定义处理:

\begin{lstlisting}[language=Rust,caption={信号处理设置}, label={lst:sys-sigaction}]
pub fn sys_sigaction(
    signum: i32,
    action: *const SignalAction,
    old_action: *mut SignalAction,
) -> isize {
    let token = current_user_token();
    let process = current_process().unwrap();
    let mut inner = process.inner_exclusive_access();
    if signum as usize > MAX_SIG {
        return -1;
    }
    if let Some(flag) = SignalFlags::from_bits(1 << signum) {
        if check_sigaction_error(flag, action as usize, old_action as usize) {
            return -1; // 禁止对SIGKILL和SIGSTOP设置处理
        }
        let prev_action = inner.signal_actions.table[signum as usize];
        *translated_refmut(token, old_action) = prev_action;
        inner.signal_actions.table[signum as usize] = *translated_ref(token, action);
        0
    } else {
        -1
    }
}
\end{lstlisting}

\subsection{信号屏蔽控制}

\texttt{sys\_sigprocmask}函数允许进程控制信号的接收状态:

\begin{lstlisting}[language=Rust,caption={信号屏蔽设置}, label={lst:sys-sigprocmask}]
pub fn sys_sigprocmask(mask: u32) -> isize {
    if let Some(process) = current_process() {
        let mut inner = process.inner_exclusive_access();
        let old_mask = inner.signal_mask;
        if let Some(flag) = SignalFlags::from_bits(mask) {
            inner.signal_mask = flag;
            old_mask.bits() as isize
        } else {
            -1
        }
    } else {
        -1
    }
}
\end{lstlisting}

\subsection{信号处理返回}

\texttt{sys\_sigreturn}函数用于从用户信号处理程序返回到正常执行流程:

\begin{lstlisting}[language=Rust,caption={信号处理返回}, label={lst:sys-sigreturn}]
pub fn sys_sigreturn() -> isize {
    if let Some(process) = current_process() {
        let mut inner = process.inner_exclusive_access();
        inner.handling_sig = -1; // 清除信号处理状态
        let trap_ctx = inner.trap_cx();
        *trap_ctx = inner.trap_ctx_backup.unwrap(); // 恢复上下文
        trap_ctx.x[10] as isize
    } else {
        -1
    }
}
\end{lstlisting}

\section{用户态系统调用封装}

用户态系统调用库为应用程序提供了便利的系统调用接口,隐藏了底层的汇编调用细节,
提供了类型安全的Rust函数接口。

\subsection{底层系统调用机制}

用户态封装通过内联汇编实现系统调用的底层机制,将系统调用号和参数正确地传递给内核:

\begin{lstlisting}[language=Rust,caption={底层系统调用实现}, label={lst:syscall-asm}]
fn syscall(id: usize, args: [usize; 3]) -> isize {
    let mut ret: isize;
    unsafe {
        asm!(
            "ecall",
            inlateout("x10") args[0] => ret,
            in("x11") args[1],
            in("x12") args[2],
            in("x17") id
        );
    }
    ret
}
\end{lstlisting}

这个函数将系统调用号存储在\texttt{x17}寄存器,参数存储在\texttt{x10-x12}寄存器,
通过\texttt{ecall}指令触发系统调用,返回值通过\texttt{x10}寄存器获取。

\subsection{文件操作接口}

文件操作函数提供了标准的文件I/O接口,支持字符串路径、字节数组缓冲区等常用数据类型:

\begin{lstlisting}[language=Rust,caption={用户态文件操作}, label={lst:user-file-ops}]
pub fn open(path: &str, flags: OpenFlags) -> isize {
    sys_open(path, flags.bits())
}

pub fn read(fd: usize, buf: &mut [u8]) -> isize {
    sys_read(fd, buf)
}

pub fn write(fd: usize, buf: &[u8]) -> isize {
    sys_write(fd, buf)
}

pub fn close(fd: usize) -> isize {
    sys_close(fd)
}

pub fn pipe(pipe_fd: &mut [usize]) -> isize {
    sys_pipe(pipe_fd)
}
\end{lstlisting}

\subsection{进程管理接口}

进程管理函数实现了标准的进程控制接口,为多进程应用程序提供了完整的开发支撑:

\begin{lstlisting}[language=Rust,caption={用户态进程管理}, label={lst:user-process-ops}]
pub fn fork() -> isize {
    sys_fork()
}

pub fn exec(path: &str, args: &[*const u8]) -> isize {
    sys_exec(path, args)
}

pub fn wait(exit_code: &mut i32) -> isize {
    loop {
        match sys_waitpid(-1, exit_code as *mut _) {
            -2 => { yield_(); }
            pid => return pid,
        }
    }
}

pub fn waitpid(pid: usize, exit_code: &mut i32) -> isize {
    loop {
        match sys_waitpid(pid as isize, exit_code as *mut _) {
            -2 => { yield_(); }
            pid => return pid,
        }
    }
}

pub fn exit(exit_code: i32) -> ! {
    sys_exit(exit_code);
    panic!("sys_exit never returns!");
}
\end{lstlisting}

\subsection{信号处理接口}

信号处理函数提供了完整的信号处理能力,支持信号发送、处理程序安装、信号屏蔽等高级功能:

\begin{lstlisting}[language=Rust,caption={用户态信号处理}, label={lst:user-signal-ops}]
#[repr(C)]
pub struct SignalAction {
    pub handler: usize,
    pub mask: SignalFlags,
}

pub fn kill(pid: usize, signum: i32) -> isize {
    sys_kill(pid, signum)
}

pub fn sigaction(
    signum: i32,
    action: Option<&SignalAction>,
    old_action: Option<&mut SignalAction>,
) -> isize {
    sys_sigaction(
        signum,
        action.map_or(core::ptr::null(), |a| a),
        old_action.map_or(core::ptr::null_mut(), |a| a),
    )
}

pub fn sigprocmask(mask: u32) -> isize {
    sys_sigprocmask(mask)
}

pub fn sigreturn() -> isize {
    sys_sigreturn()
}
\end{lstlisting}

这样使得用户程序可以使用类型安全的Rust接口进行系统调用,
而底层的参数转换和错误处理对应用程序完全透明,也可以简化了用户程序的开发复杂度。