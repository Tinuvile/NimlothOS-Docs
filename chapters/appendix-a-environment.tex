\chapter{环境配置}

本附录主要介绍构建和运行NimlothOS所需的开发环境配置。为了确保项目能够正常编译和运行,请按照以下步骤配置环境。

\section{系统要求}

推荐使用Linux系统,Windows用户可以通过WSL2获得良好体验。

\textbf{操作系统要求:}
\begin{itemize}
    \item Linux发行版:Ubuntu 20.04+
    \item Windows:通过WSL2环境支持,推荐使用Ubuntu 20.04 LTS子系统
\end{itemize}

\section{Rust工具链配置}

NimlothOS使用Rust语言开发,需要配置特定版本的Rust工具链以确保编译的一致性和稳定性。

\subsection{Rust版本要求}

项目使用固定的Rust nightly版本以确保构建的可重现性:

\begin{lstlisting}[language=bash]
# 安装指定的Rust nightly版本
rustup install nightly-2025-02-18
rustup default nightly-2025-02-18
# 验证Rust版本
rustc --version
# 应显示:rustc 1.77.0-nightly (hash 2025-02-18)
\end{lstlisting}

选择nightly版本是因为NimlothOS需要使用一些实验性的Rust特性,如内联汇编、无标准库开发等功能。

\subsection{目标架构配置}

添加RISC-V 64位目标架构支持:

\begin{lstlisting}[language=bash]
# 添加RISC-V 64位目标架构
rustup target add riscv64gc-unknown-none-elf
# 验证目标架构安装
rustup target list --installed | grep riscv64
\end{lstlisting}

\texttt{riscv64gc-unknown-none-elf}目标架构表示:
\begin{itemize}
    \item \texttt{riscv64}:RISC-V 64位架构
    \item \texttt{gc}:包含通用扩展(G)和压缩指令扩展(C)
    \item \texttt{unknown}:未指定供应商
    \item \texttt{none}:无操作系统环境(裸机开发)
    \item \texttt{elf}:使用ELF二进制格式
\end{itemize}

\subsection{组件安装}

安装Rust开发所需的一些组件:

\begin{lstlisting}[language=bash]
# 安装Rust源码(用于构建核心库)
rustup component add rust-src
# 安装LLVM工具(用于二进制文件操作)
rustup component add llvm-tools-preview
\end{lstlisting}

\subsection{Cargo扩展工具}

安装用于二进制文件操作的Cargo扩展:

\begin{lstlisting}[language=bash]
# 安装cargo-binutils
cargo install cargo-binutils
# 验证安装
cargo objdump --version
cargo objcopy --version
\end{lstlisting}

\texttt{cargo-binutils}提供了Rust版本的二进制工具,包括:
\begin{itemize}
    \item \texttt{cargo objdump}:反汇编工具
    \item \texttt{cargo objcopy}:二进制格式转换工具
    \item \texttt{cargo size}:分析二进制文件大小
    \item \texttt{cargo nm}:符号表查看工具
\end{itemize}

\section{QEMU虚拟机配置}

QEMU是运行NimlothOS的核心虚拟化环境,提供了完整的RISC-V硬件模拟支持。

\subsection{QEMU安装}

\begin{lstlisting}[language=bash]
# Ubuntu安装
sudo apt update
sudo apt install qemu-system-misc
# 验证安装和版本
qemu-system-riscv64 --version
# 推荐版本:QEMU emulator version 7.0+
\end{lstlisting}

\subsection{版本要求说明}

NimlothOS要求QEMU版本不低于7.0,当前测试版本为7.2.0。

\subsection{QEMU配置验证}

验证QEMU的RISC-V支持:

\begin{lstlisting}[language=bash]
# 检查支持的机器类型
qemu-system-riscv64 -machine help | grep virt
# 检查支持的CPU类型
qemu-system-riscv64 -cpu help
\end{lstlisting}

\section{RISC-V交叉编译工具链}

虽然NimlothOS主要使用Rust开发,但某些调试和分析任务仍需要传统的GCC工具链支持。

\subsection{GCC工具链安装}

安装RISC-V交叉编译工具链:

\begin{lstlisting}[language=bash]
# Ubuntu/Debian系统
sudo apt install gcc-riscv64-unknown-elf gdb-multiarch
# 验证安装
riscv64-unknown-elf-gcc --version
# 推荐版本:9.3.0+
riscv64-unknown-elf-gdb --version
\end{lstlisting}

对于其他发行版,可能需要从源码编译或使用第三方仓库。

\section{环境验证}

完成所有配置后,通过以下步骤验证环境是否正确配置:

\subsection{工具链验证}

\begin{lstlisting}[language=bash]
# 验证Rust工具链
rustc --version | grep nightly-2025-02-18
rustup target list --installed | grep riscv64gc-unknown-none-elf
cargo objdump --version
# 验证QEMU
qemu-system-riscv64 --version | grep "version 7"
# 验证GCC工具链
riscv64-unknown-elf-gcc --version
\end{lstlisting}

\subsection{项目构建测试}

克隆并构建NimlothOS项目进行最终验证:

\begin{lstlisting}[language=bash]
# 克隆项目
git clone https://github.com/Tinuvile/NimlothOS.git
cd NimlothOS
# 构建内核
cd os
make build
# 运行系统
make run
\end{lstlisting}
