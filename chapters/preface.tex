\chapter*{前言}
\addcontentsline{toc}{chapter}{前言}

学习操作系统以来,我就一直想写一个自己的OS,因此本次课设我的选题也毫无疑问就是这个。我最先看的是
\href{https://os.phil-opp.com/}{Writing an OS in Rust},这是一个在x86上构建的操作系统,
但作者仅仅写到地址分页部分就戛然而止,考虑到暑假时间有限,加上我对x86的熟悉程度不如RISC-V,我决定跟着
\href{https://rcore-os.cn/rCore-Tutorial-Book-v3/index.html}{rCore-Tutorial-Book-v3 3.6.0-alpha.1}
教程重新开始。

rCore教程是一个从零开始搭建操作系统的教程,它把一个功能较为完整的操作系统拆解成了逐步的各个小型操作系统,
这也是我在暑假阶段主要完成的内容,我跟着教程依次完成了以下各个部分的实现。代码和提交记录参见
\href{https://github.com/Tinuvile/NimlothOS}{github}上的各个分支。

\begin{itemize}
    \item \textbf{LibOS}:让应用与硬件隔离
    \item \textbf{BatchOS}:让应用与OS隔离
    \item \textbf{Multiprog\&TimesharingOS}:允许多应用、分时共享CPU资源
    \item \textbf{AddressSpaceOS}:隔离应用地址空间与内核地址空间
    \item \textbf{ProcessOS}:支持应用动态创建新进程,添加进程和资源管理能力
    \item \textbf{FileSystemOS}:添加文件系统,支持应用数据的持久化保存
    \item \textbf{IPCOS}:实现进程间通信与I/O重定向
    % \item \textbf{Thread\&CoroutineOS}:完善并发功能,支持线程和协程
    % \item \textbf{SyncMutexOS}:在多线程APP中支持对共享资源的同步互斥访问
    % \item \textbf{DeviceOS}:支持基于外设中断的串口、块设备、键盘、鼠标、显示设备    
\end{itemize}

当然,目前的操作系统仍有很多不足,我希望在未来继续完善它。比如添加线程管理机制、完善I/O设备管理机制和添加更多的设备驱动等。

同时我也很喜欢模块化设计,在暑假的学习过程中,我发现了\href{https://github.com/unikraft/unikraft}{unikraft}
和\href{https://github.com/arceos-org/arceos}{arceos}这样优秀开源的模块化操作系统,
它们将是我未来学习的对象。

\section*{致谢}

在操作系统的学习旅途中,非常感谢王冬青、张惠娟两位老师对我的教学与帮助,同时也衷心感谢rCore、xv6、blogOS等优秀开源项目和操作系统教程,
希望他们越来越好。