\chapter{系统启动与初始化}

我将本系统的启动与初始化分为三个阶段:

\begin{itemize}
    \item \textbf{QEMU初始化}
    \item \textbf{BootLoader初始化}
    \item \textbf{内核初始化}
\end{itemize}

\section{QEMU初始化}

这个阶段在实际启动中是ROM段。机器上电、文件加载到内存后,QEMU CPU的PC会被设置为0x1000,然后执行第一阶段的代码。
它会对CPU进行一些初始化操作,然后把控制权移交给bootloader,它的跳转地址固定是0x80000000处,rustsbi-qemu.bin被预先加载到这里。

\section{BootLoader初始化}

在实际计算机中,Loader会进行内存初始化,并加载Runtime和BootLoader。BIOS和UEFI就是x86\_64架构的Loader。


但在NimlothOS中,它们在QEMU初始化阶段便已经被加载到了内存中,跳转以后即可直接进入BootLoader部分。BootLoader在实际电脑中会进行OS镜像的加载,但
NimlothOS的镜像是在QEMU启动前便加载到物理内存中了,因此它只需要作为RunTime并跳转到内核即可。

Runtime固件程序为OS提供运行时服务,它是对硬件最基础的抽象,\href{https://github.com/riscv-non-isa/riscv-sbi-doc}{SBI}
就是RISC-V架构下的Runtime规范,RustSBI的初始化结束以后,会跳转到预设好的0x80200000,这里放有内核镜像。

\section{内核初始化}

第三阶段便是内核控制的操作系统各个模块的初始化,下面进行详细介绍。

\subsection{BSS段清零}

\subsection{日志系统初始化}

\subsection{内存系统初始化}

\subsection{用户初始进程注册}

\subsection{启用时钟中断}

\subsection{进入主循环调度}